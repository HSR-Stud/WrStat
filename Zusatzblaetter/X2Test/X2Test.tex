%Autor: Simon Walker
%Version: 1.0
%Datum: 26.12.2019
%Lizenz CC-BY-NC-SA
\newcommand{\titleinfo}{$\chi^{2}$-Test}
\newcommand{\authorinfo}{Simon Walker}  

% Genereller Header
\documentclass[10pt,a4paper,fleqn]{article}
% Dateiencoding
\usepackage[utf8]{inputenc}
% Seitenränder
\usepackage[left=2cm,right=2cm,top=1cm,bottom=1cm,includeheadfoot]{geometry}
% Sprachpaket
\usepackage[ngerman]{babel,varioref}

\usepackage{amsmath}

\usepackage{tabularx}
\newcolumntype{Y}{>{\centering\arraybackslash}X}
\newcommand{\Fett}[1]{\large{$\mathbf{#1}$}}

\begin{document}
	\pagestyle{empty}
	\huge{\textbf{$\mathbf{\chi^{2}}$-Test}}\\[10pt]
	\normalsize
	\begin{tabularx}{\textwidth}{Xrp{180pt}}
		& \Large{Name: } & \hrule \\
		&&\\
		& Dieses Blatt gehört zur Aufgabe: & \hrule \\
	\end{tabularx}
	\\[20pt]
	
	\renewcommand{\arraystretch}{1.2}
	\begin{tabularx}{\textwidth}{p{70pt}X}
		\large{Nullhypothese:} & \hrule\\
		& \hrule\\
	\end{tabularx}
	
	\begin{tabularx}{\textwidth}{p{70pt} X r X r X}
		\large{Gewähltes $\alpha$:} & \hrule &
		\large{Anzahl Freiheitsgrade:} & \hrule &
		\large{$D$-Krit:} & \hrule
	\end{tabularx}
	

	\renewcommand{\arraystretch}{2}
	\begin{tabularx}{\textwidth}{|c|Y|Y|Y|Y|Y|}
		\hline
		\Fett{i} & \large{\textbf{Ausgang}} & \Fett{p_i} & \Fett{n_i} & \Fett{n_i-np_i} & \Fett{(n_i-np_i)^2/np_i}\\ \hline \hline
		\Fett{1} &     &       &     &       &     \\ \hline
		\Fett{2} &     &       &     &       &     \\ \hline
		\Fett{3} &     &       &     &       &     \\ \hline
		\Fett{4} &     &       &     &       &     \\ \hline
		\Fett{5} &     &       &     &       &     \\ \hline
		\Fett{6} &     &       &     &       &     \\ \hline
		\Fett{7} &     &       &     &       &     \\ \hline
		\Fett{8} &     &       &     &       &     \\ \hline
		\Fett{9} &     &       &     &       &     \\ \hline
		\Fett{10}&     &       &     &       &     \\ \hline
		         &     &       &$n=\sum$ &    &$D = \sum$\\ \hline
		         &     &       &     &       &     \\ \hline
		
	\end{tabularx}\\[25pt]
	
	\renewcommand{\arraystretch}{1.2}
	\begin{tabularx}{\textwidth}{r X}
		\large{Schlussfolgerung:} & \hrule\\
		& \hrule\\
		& \hrule\\
	\end{tabularx}	
	
\end{document}
