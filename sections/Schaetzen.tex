\newpage
\section{Schätzen \skript{\pageref{sk-chapter-schaetzen}}}

	\subsection{Konsistente Schätzer \skript{\pageref{sk-section-konsistente-schaetzer}}}
		Ein Schätzer ist konsistent, wenn $\lim \limits_{n \rightarrow \infty}$ = E(X)
		ergibt\\
		\begin{tabular}{p{10cm}p{8cm}}
        Der Mittelwert der Stichprobe ist ein konsistenter Schätzer.
        & $\lim\limits_{n\to\infty}=\frac{X_1+\ldots+X_n}{n}=E(X)$
        \end{tabular}

        \hspace*{2.1mm}Der Schätzer $\bar{X}=\frac{X_1+\ldots +X_n}{n}$ heisst
        der Stichprobenmittelwert der Stichprobe $X_1,\ldots,X_n$. \\        


	\subsection{Erwartungstreue Schätzer \skript{\pageref{sk-section-erwartungstreue-schaetzer}}}
	\label{Stichprobenvarianz}
		Ein Schätzer ist erwartungstreu, wenn $E($Schätzer$)=E($realer Wert$)$\\
		\begin{tabular}{p{8cm}p{10cm}}
        Ist der Stichprobenmittelwert ein konsistenter Schätzer, aber er ist
        sogar erwartungstreu:
        & $E(\mu(X_1,\ldots,X_n))=\frac{E(X_1)+\ldots+E(X_n)}{n}=E(X)$\\
        Erwartungstreue Schätzer für $var(x)$ ist:\\
       \fbox{$S^2=\frac{n}{n-1}(\underbrace{\frac{1}{n}\sum X_i^2}_{E(X^2)}-
               \underbrace{(\frac{1}{n}\sum X_i)^2)}_{E(X)^2}$}
        & Stichprobenvarianz, empirische Varianz\\
        $S^2=\frac{1}{n-1}\sum\limits_{i=1}^n(X_i-\bar{X})^2$
        & $\bar{X}=M_n$ heisst Stichprobenmittelwert\\
        \end{tabular}
	\subsubsection{Kleinstmöglicher Fehler}
		$E( (E(X)- \frac{x_1+\ldots+x_n}{n})^2)= minimal$ \\

\hrule

	\subsection{Maximum Likelihood Schätzer \skript{\pageref{sk-section-maximum-likelihood-schaetzer}}}
	Sinn des Likelihoodschäzers ist einen unbekannten Parameter $\vartheta$ einer Dichtefunktion
	$\phi(x, \vartheta)$ zu schätzen.
	
	$$L(x_1,\ldots,x_n;\vartheta)=\phi(x_1,\vartheta)\cdot\ldots\cdot\phi(x_n,\vartheta) \quad \Longrightarrow \quad
	\frac{d}{d \vartheta} L(x_1,\ldots,x_n;\vartheta) = 0 \quad \Longrightarrow \quad \vartheta = ? 
	\text{	(Maximum-Likelihood-Schätzer})$$
	
	Für eine normalverteilte Grösse lautet die Likelihood Funktion:
	$L(x_1,\ldots,x_n;\vartheta)=\frac{1}{(\sqrt2\pi)^n}e^{-\frac{1}{2\sigma^2}\sum\limits_{i=1}^n (x_i-\vartheta)^2}$\ 

	Der unbekannte Parameter $\vartheta$ kann nun durch suchen des Maximums der Funktion ermittelt
	werden ($\vartheta$ wird variert). Die Funktion wird maximal, wenn die Summe im
	Exponent minimal wird. Das $\vartheta$, das die Summe minimiert, kann durch
	\textbf{ableiten nach $\vartheta$ und null setzen ermittelt} werden. Es können
	auch Stichprobenvarianz $S^2$ oder Ähnliches ermittelt werden. \\
	
\hrule

  \subsection{Verteilung der Schätzwerte \skript{\pageref{sk-section-verteilung-der-schaetzwerte}}}
    $X_1, \ldots, X_n$ sind unabhängige, normalverteilte Zufallsvariablen mit Erwartungswert
    $\mu$ und Varianz $\sigma^2$. Dann gilt
    \begin{enumerate}
      \item $\bar{X}$ und $S^2$ sind unabhängig
      \item $\bar{X} = \frac{X_1 + \ldots X_n}{n}$ ist normalverteilt mit $E(\bar{X}) = \mu$
            und $var(\bar{X}) = \frac{\sigma^2}{2}$
      \item \fbox{$\frac{n-1}{\sigma^2} \cdot S^2$ ist $\chi_{n-1}^2$-verteilt} mit
            $S^2 = \frac{1}{n-1} \sum\limits_{i=1}^n (X_i - \bar{X})^2$
    \end{enumerate}
    
\newpage

	\subsection{Konfidenzintervall von Messwerten}
		\subsubsection{Konfidenzintervall \skript{\pageref{sk-section-konfidenzintervalle}}}
			\begin{tabular}{p{18cm}}
		     Ein Intervall $[L(X_1,\ldots,X_n),R(X_1,\ldots,X_n)]$ heisst ein
		     $1-\alpha$- Konfidenzintervall für den Parameter $\vartheta$, wenn der wahre
		     Wert des Parameters $\vartheta$ höchstens mit Wahrscheinlichkeit $\alpha$
		     ausserhalb des Intervalls liegt.\\
		     Es gilt: $P(L \leq \vartheta \leq R) = 1-\alpha$
		    \end{tabular}\\
		    
	   
    \subsubsection{Bei bekannter Varianz $\sigma^2$}
     Falls Varianz $\sigma^2$ von Messwerten bekannt ist, handelt es sich bei $\bar{X}$ um \textbf{normalverteilte} Zufallsvariable mit Varianz $\sigma^2/n$. \\
    \begin{tabular}{p{8cm}p{8cm}}
    Also kann sehr einfach ein x für das Konfidenzintervall gefunden werden:
    &$P\left(\left|\frac{\bar{X}-\mu}{\sigma / \sqrt{n}}\right|\leq x\right) = 1 - \alpha \Rightarrow F(x) = 1- \frac{\alpha}{2}$\\
    Daraus ergibt sich folgendes Konfidenzintervall
    &$\mu\in\left[\bar{X}-x\frac{\sigma}{\sqrt{n}},\bar{X}+x\frac{\sigma}{\sqrt{n}}\right]$ mit W'keit $1-\alpha$\\
    \end{tabular}\\
    
    \subsubsection{Bei geschätzter Varianz $S^2$}
    \textbf{t-Verteilung}\\
    	Der Mittelwert ($\frac{x_1+\ldots+x_n}{n}$) normalverteilter Daten ist
        t-Verteilt, wenn Varianz mit Stichprobenvarianz geschätzt wurde.\\
        Ab einer gewissen Anzahl Messungen ($n \geq 30$) kann näherungsweise auch mit
        der Normalverteilung gerechnet werden.  \\
%    	\begin{tabular}{p{8cm}p{8cm}}
%        Die Wahrscheinlichkeitdichte der
%        t-Verteilung ist: &$\varphi_t(t)=\frac{\Gamma (\frac{k+1}{2})}{\sqrt{\pi
%        k}\Gamma(\frac{k}{2})}\left(1+\frac{t^2}{k}\right)^{- \frac{k+1}{2}}$\\ \\
%        \end{tabular}\\
    
	\begin{minipage}{11cm}
 		\textbf{Checkliste}\\
		\begin{tabular}{ll}
        1) $\bar{X}, S$ als Schätzungen aus $x_i$ bestimmen\\
        2) $t$ aus {\em t-Tabelle} $(k=n-1)$ für $1- \dfrac{\alpha}{2}$ = W'keit für eine Seite\\
        3) Intervall
        $\left[\bar{X}-t\frac{S}{\sqrt{n}},\bar{X}+t\frac{S}{\sqrt{n}}\right]$,
        $(1-\alpha)$ Konfidenzintervall
        \end{tabular}\\
		\textbf{Anwendung}\\
		\begin{tabular}{ll}
        $\frac{\textcolor{red}{\bar{X}}-\mu}{\textcolor{blue}{S}/\sqrt{\textcolor{green}{n}}}$
        & t-Verteilt\\ \\
        \end{tabular}
    
    \end{minipage}
	\begin{minipage}{10cm}
   		\includegraphics[width=8cm,height=4cm]{./bilder/T-Verteilung.png}\\
		$\lim\limits_{x\rightarrow \infty}$ = 0 aber langsamer wie bei
		Gaussverteilung 
    \end{minipage}

		\begin{tabular}{p{2cm}p{16cm}}
        Beispiel: & \textcolor{green}{10} Messungen ergeben Durchschnittswert
        \textcolor{red}{4{,}7} und eine Standardabweichung \textcolor{blue}{0{,}1}.
        Finde ein 99\%  Konfidenzintervall für $\mu$.\\
        
         Finde t: & \begin{tabular}{| c | c | c | c |}
                   \hline
                   $k=\textcolor{green}{n}-1$ & \ldots & 0{,}995\\
                   \hline
                   \vdots & \vdots & \vdots \\
                   \hline
                   9 & \ldots & 3{,}2498\\
                   \hline
                   \end{tabular}
        
       		 $\left[\textcolor{red}{\bar{X}}-3,2498\frac{\textcolor{blue}{S}}{\sqrt{\textcolor{green}{n}}},
		\textcolor{red}{\bar{X}}+3,2498\frac{\textcolor{blue}{S}}{\sqrt{\textcolor{green}{n}}}\right]
		\Rightarrow 
		\left[{\color{red}4{,}7}-3{,}2498\frac{\textcolor{blue}{0{,}1}}{\sqrt{\textcolor{green}{10}}},
		\textcolor{red}{4{,}7}+3{,}2498\frac{\textcolor{blue}{0{,}1}}{\sqrt{\textcolor{green}{10}}}\right]$\\ \\
		& $\mu\in \left[4{,}5072, 4{,}8028\right]$ mit Wahrscheinlichkeit 99\%
        \end{tabular}