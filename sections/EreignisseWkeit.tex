\section{Ereignisse und ihre Wahrscheinlichkeit}

\subsection{Kombinatorik \skript{5} \sachs{66}}
\begin{tabular}{| p{5.5cm} | c | c |}
  \hline
  Art der Auswahl bzw. Zusammenstellung von $k$ aus $n$ Elementen &
  \multicolumn{2}{|c|}{Anzahl der Möglichkeiten} \\
  & ohne Wiederholungen & mit Wiederholungen \\
 	&	$(k\leq n)$         & $(k\leq n)$ \\
 	\hline
 	Permutationen & $P_n=n!\quad(n=k)$       & $P_n^{(k)}=\frac{n!}{k!}$ \\
 	Kombinationen & $C_n^{(k)}=\binom n k$   & $C_n^{(k)}=\binom{n+k-1} k$ \\
  Variationen   & $V_n^{(k)}=k!\binom n k$ & $V_n^{(k)}=n^k$\\
  \hline
\end{tabular}
\begin{list}{$\bullet$}{\setlength{\itemsep}{0cm} \setlength{\parsep}{0cm}
\setlength{\topsep}{0.1cm}}
  \item \textbf{Permutationen}: Gegeben seien $n$ verschiedene Objekte. Dann
    gibt es $n!$ verschiedene Reihenfolgen in denen man diese Objekte anordnen
    kann. \\
    z.B.: $x,y,z;\quad x,z,y;\quad z,y,x;\ldots$
  \item \textbf{Kombination}: Gegeben seien $n$ verschiedene Objekte. Dann gibt
    es $\binom n k$ Möglichkeiten, daraus $k$ Objekte auszuwählen, wenn es nicht
    auf die Reihenfolge ankommt. \\
    z.B.: Wie viele verschiedene Möglichkeiten hat man beim Lotto, 6 Zahlen aus
    49 auszuwählen?
  \item \textbf{Variation} nennt man eine Auswahl von $k$ Elementen aus $n$
    verschiedenen Elementen unter Beachtung der Reihenfolge
\end{list}
\hrule

\subsection{Wahrscheinlichkeit \& Rechenregeln \skript{32}}
\begin{tabular}{ll}
  Wertebereich:         & ${0}\le{P(A)}\le{1}$\\
  Sicheres Ereignis:    & $P(\Omega)=1$\\
  unmögliches Ereignis: & $P(\emptyset)=0$
\end{tabular}
		\begin{tabular}{ll}
			komplementär Ereignis:
			&$P(\bar{A})=P({\Omega}\setminus{A})=1-P(A)$\\
			Differenz der Ereignisse A und B:
			&$P({A}\setminus{B})=P(A)-P({A}\cap{B})$\\
			Vereinigung zweier Ereignisse:
			&$P({A}\cup{B})=P(A)+P(B)-P({A}\cap{B})$
		\end{tabular}
\hrule

\subsection{Unabhängige Ereignise \skript{36} \sachs{83}}
		Unabhängige Ereignisse $A$ und $B$ liegen vor, wenn: \hspace*{5mm} $P(A\mid
		B)=P(A)$ \hspace{4mm} und \hspace{4mm} $P(B\mid A)=P(B)$ \hspace*{5mm} erfüllt
		ist. \\
		Für sie gilt \hspace*{5mm} $P(A\cap B)=P(A)P(B)$\\
    	Die Tatsache, dass A eingetreten ist, hat keinen Einfluss auf die 
		Wahrscheinlichkeit von B.\\
\hrule

\subsection{Bedingte Wahrscheinlichkeit \skript{35} \sachs{85}}
Die Wahrscheinlichkeit für das Eintreten des Ereignisses $A$ unter der
Bedingung, dass das Ereignis $B$ bereits eingetreten ist.
\begin{center}
  $P(A\mid B)= \dfrac{P(A\cap B)}{P(B)}=\underbrace{\frac{P(A)\cdot
  (B)}{P(B)}=P(A)}_{\text{nur wenn unabhängig}} \qquad P(\overline{A}|B) = 1 -
  P(A|B)$
\end{center}
\hrule

\begin{minipage}{7cm}
\vspace{3mm}
\subsection{Satz von Bayes \skript{43} \sachs{89}}
\begin{tabular}{ll}
  $P(B\mid A)=P(A\mid B) \cdot\dfrac{P(B)}{P(A)}$
  \vspace{1mm}
\end{tabular}
\hrule

\subsection{Laplace-Ereignisse \skript{47} \sachs{79}}
In einem endlichen Wahrscheinlichkeitsraum $\Omega$ haben alle
Elementarereignisse die gleiche Wahrscheinlichkeit. \\
$P(A)=\dfrac{\left| A\right|}{\left|\Omega\right|}$ \\
\hrule

\end{minipage}
\hspace{5mm}
\begin{minipage}{12cm}
\subsection{Totale Wahrscheinlichkeit \skript{58} \sachs{88}}
	$P(A)=\sum\limits_{i=1}^N P(A\mid G_i)\cdot P(G_i)$ \\
	
	in Matrixform: \\
	$\begin{pmatrix}P(A_1)\\P(A_2)\\\vdots\\P(A_n)\end{pmatrix} = 
	\underbrace{\begin{pmatrix}P(A_1|B_1) & P(A_1|B_2) & \ldots & P(A_1|B_n) \\
	P(A_2|B_1) & P(A_2|B_2) & \ldots & P(A_2|B_n) \\
	\vdots & \vdots & \ddots & \cdots \\
	P(A_m|B_1) & P(A_m|B_2) & \ldots & P(A_m|B_n)\end{pmatrix}}_{\text{W'keitsmatrix}}
	\cdot \begin{pmatrix}P(B_1)\\P(B_2)\\\vdots\\P(B_n)\end{pmatrix}$
\end{minipage}