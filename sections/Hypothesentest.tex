\newpage
\section{Hypothesentest}
	\subsection{Grundsätze}
	- ``Man braucht Aussagen, die man widerlegen könnte.''\\
	- Irrtum ist möglich (Irrtumsw'keit $\alpha$)\\
	- Beweis durch Widerlegen des Gegenbeweises
	\subsection{Vorgehen (Nullhypothese)}
	\begin{enumerate}
      \item Hypothese, die der Test \textbf{widerlegen} soll\\
      	$\rightarrow$ Nullhypothese $\Rightarrow$ Keine Wirkung/Effekt
      \item Festlegung der Irrtumswahrscheinlichkeit 	$\alpha$ = 0.05, 0.01, \ldots (Niveau=1-$\alpha$)
      \item Testgrösse X, W'keitsverteilung
      	$\rightarrow$ Nur Werte bis zum getesteten Ereignis betrachten! (Nicht Zukunft mit einbeziehen)
      \item Bestimmung der Schranken $x_{krit}$ für: 		
      	\begin{itemize}
      		\item Einseitiger Test $P(X > x_{krit})=\alpha$
      		\item Zweiseitiger Test $P(|X| > x_{krit})=\alpha/2$
      	\end{itemize}
      \item Wert für $F\!\left({\frac{x_{krit}-\mu}{\sigma}}\right)$ aus Tabelle \ref{tabelle-quantilen-normalverteilung} (S. \pageref{tabelle-quantilen-normalverteilung})
      \item Falls Messungen ergeben $X > x_{krit} \Longrightarrow$ Hypothese
      \textbf{falsch} mit W'keit $1-\alpha$
    \end{enumerate}
    

	\subsection{Testen einer diskreten Verteilung}
	\subsubsection{$\chi^2$-Test mit k-möglichen Ausgängen}
		
		\begin{tabular}{l l}
			Mögliche Ausgänge: $I_i$, $i=1,\ldots,k$&
			Wahrscheinlichkeit von Ausgang $i$: $P(X\in I_i)=\textcolor{red}{p_i}$\\
			$n$ Beobachtungen, davon jeweils \textcolor{blue}{$n_i$} mit Ausgang $i$ &
			$\boxed{D=\sum\limits_{i=1}^{k}\frac{(\textcolor{blue}{n_i}-\textcolor{red}{np_i})^2}{\textcolor{red}{np_i}}}$ \hspace{4mm} ist $\chi^2_{k-1}$, mit $k-1$ Freiheitsgrade\\
		\end{tabular}
        
       % $\varphi_n(x)=  \begin{cases}\displaystyle \frac{x^{\frac{r}{2}-1}e^{
        %-\frac{x}{2}}}{2^{\frac{r}{2}}\Gamma\left(\frac{r}{2}\right)} & x>0 \\
        %0 & x\leq 0 \end{cases} $\\  
		%$E(X)=r; \quad var(X)=2r$

		\textbf{Durchführung des $\chi^2$-Tests}\\
		\begin{minipage}{13cm}
		\begin{tabular}{p{4cm}p{8cm}}
		1. Daten erfassen: &
			Damit der Test optimal funktioniert muss folgende Bedingung erfüllt sein:
			$\boldsymbol{n_i\geq 5 \; \forall \; i}$\\
        
        2. \textbf{Diskrepanz $D$} berechnen \scriptsize
        Vorlage auf \href{https://github.com/RostBau/WrStat/tree/master/Zusatzblaetter/X2Test}{GitHub} %TODO Link überprüfen
        \normalsize& 
        \begin{tabular}[t]{|c|c|c|c|c|c|}
                                  \hline
                                  $i$ & Ausgang & $p_i$ & $n_i$ & $(n_i-np_i)$ &
                                  $(n_i-np_i)^2/np_i$\\
                                  \hline
                                  1 & & & & & \\
                                  2 & & & & & \\
                                  3 & & & & & \\
                                  4 & & & & & \\
                                  5 & & & & & \\
                                  \hline
                                  & & & n & & $D=\sum$ \\
                                  \hline
                                  
                                  \end{tabular}\\
        3. Schwellenwert für $D$ & $x_{1-\alpha}$ für $F_{\chi_{k-1}^2}(x) = 1- \alpha$ aus $\chi_{k-1}^2$-Tabelle lesen. \\
        & Wenn
        $\alpha$ nicht vorgegeben, dann $\alpha$ z.B. $0.1$ oder $0.05$
        wählen.\\
        &Anzahl Freiheitsgrade= Anzahl Ausgänge$ -1 $ = $k - 1$ \\ & $p=1-\alpha$ \hspace{10mm}$\alpha=$Irrtumswahrscheinlichkeit\\
        & \textcolor{green}{$D \geq x_{1-\alpha}$} Hypothese ist unwahrscheinlich!\\
        & $D < x_{1-\alpha}$ Hypothese nicht wiederlegbar!\\
        & \textcolor{blue}{$D \leq x_\alpha$} Daten möglicherweise
        ''fabriziert''!\\
        \end{tabular}
        \end{minipage}
		\begin{minipage}{6cm}
		\vspace*{6cm}
        \includegraphics[width=6cm]{./bilder/chi-test.png}
        \end{minipage}\\
	\begin{tabular}{ll}
    	Nachteile:&	-Grob verpixeltes Bild der Verteilung\\
    	&			-wenn wenig Messwerte $\Longrightarrow$ geringe Aussagekraft
    \end{tabular}
   	
  \subsection{Testen einer stetigen Verteilung}
  	\subsubsection{Mit $\chi^2$-Test}
  	\begin{minipage}{13cm}
  		Der $\chi^2$-Test kann im Prinzip nur diskrete Wahrscheinlichkeitsverteilungen testen.\\
  		$\rightarrow$ Will man eine stetige Verteilung testen, muss man zunächst Klassen von Werten bilden. $\rightarrow$ Anschliessend deren Wahrscheinlichkeiten berechnen und dann prüfen ob die künstlich diskrete Verteilung im $\chi^2$-Test Bestand hat.\\
  		$\Rightarrow$ Achtung: Durch die Klassenbildung wird eine künstliche Diskretisierung eingeführt. 
  	\end{minipage}		
  	\begin{minipage}{6.5cm}
  	        \includegraphics[width=6cm]{./bilder/chi_quadrat.png}\\
  	        \end{minipage}
  	        
  	
	\subsubsection{Kolmogorov-Smirnov Test}
	\begin{tabular}{ll}
	\textbf{Nullhypothese} & Messwerte $x_1, ..., x_n$ haben Verteilungsfunktion $F_X$\\
    Idee: & Vergleiche Verteilungsfunktionen (statt der Dichtefunktionen)\\
    Daten: & $X$ Zufallsvariable, Verteilungsfunktion $F_X$, $n$ Messungen
    ergeben Stichprobe $x_i$
    \end{tabular}

	\begin{minipage}{6cm}
    \includegraphics[width=6cm]{./bilder/ks1.png}\\
    \includegraphics[width=6cm]{./bilder/ks2.png}\\
    \end{minipage}
	\begin{minipage}{12cm}
    \textcolor{blue}{$F_X$ theoretische Verteilungsfunktion von X}\\ \\
    \vspace{15mm}
    \textcolor{green}{empirische Verteilungsfunktion $\frac{Anzahl\{x_i \leq
    x \}}{n}$}\\

    Testgrösse\\
    \hspace*{0.5cm}
    	\parbox{6cm}{\textcolor{red}{max}
   			(\textcolor{green}{$F_{emp}$(x)} - 
   			\textcolor{blue}{$F_x$(x)})\\
    		\textcolor{red}{min} 
    		(\textcolor{green}{$F_{emp}(x)$}-
    		\textcolor{blue}{$F_x(x)$})}

    \end{minipage}

	\textbf{Durchführen des Kolmogorov-Smirnov Test}\\
	\begin{tabular}{l p{18cm}}
		1. &  Werte $x_1, ..., x_n$ in aufsteigender Reihenfolge sortieren\\
		2. & \parbox[t]{18cm}{
				$\mathbf{K_n^{\pm}}$ \textbf{berechnen}\\
				\hspace*{0.5cm}
				\parbox{6cm}{
					$\boxed{K_n^+ = \sqrt{n}\underset{1<x<n}{max}\left(\frac{i}{n} - F_X(x_i)\right)}$\\
					$\boxed{K_n^- = \sqrt{n}\underset{1<x<n}{max}\left(F_X(x_i)- \frac{i-1}{n}\right)}$}
				\parbox[c]{8cm}{
					Mit Tabelle:
					\begin{tabular}[t]{|c|c|c|c|c|}
				   		\hline
				    	$i$ & $x_i$ & $i/n$ & $F_X(x_i)$ & $(i-1)/n$ \\
				    	\hline
				    	1 & $min(x_i)$ & & & \\
				    	& $\vdots$ & & &\\	    
				    	n & $max(x_i)$ & & &\\
				     	\hline
				     
				 \end{tabular}
				 }		
		}\\
	3. & Finde $t_{n,1-\alpha}$,$t_{n,\alpha}$  in der Tabelle \ref{tabelle-quantilen-kstest} (S. \pageref{tabelle-quantilen-kstest})\\
	4. & Falls $K^+_n > t_{n,1-\alpha}$ oder $K^-_n < t_{n,\alpha}$,
			verwerfe die Hypothese, dass $X$ die Verteilungsfunktion $F_X$ hat. 
	\end{tabular}

\subsection{Vergleichen von Mittelwerten (t-Test)}
	\begin{tabular}{l l}
		\textbf{Nullhypothese} & 
		$\Rightarrow$ $E(X) = E(Y)$\\
		
		\textbf{Daten}&
		$\mathbf{n}$ Messungen von $x_i \Rightarrow \bar{X} \; ; \; S_X^2$ \\ 
		& $\mathbf{m}$ Messungen von $y_i \Rightarrow \bar{Y} \; ; \; S_Y^2$\\
		
		\textbf{Testgrösse} &
		$T = \dfrac{\bar{X}-\bar{Y}}{\sqrt{(n-1)\cdot S_X^2 + (m-1)\cdot S_Y^2}}\cdot \sqrt{\dfrac{n m (n+m-2)}{n + m}}$\\
		& T ist t-verteilt\\
		
		\textbf{Schwellenwert} &
		$t_{krit}$ aus Tabelle \ref{tabelle-quantilen-tvert} mit Freiheitsgrad $n+m-2$ und $p = 1-\alpha$\\
		
		\textbf{Test} &
		Falls $T > t_{krit}$ wird die Hypothese $E(X) = E(Y)$ verworfen
		
	\end{tabular}
	 
	
	
	\