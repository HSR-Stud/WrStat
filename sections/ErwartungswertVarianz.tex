\section{Erwartungswert und Varianz}

\subsection{Erwartungswert \skript{63}}
\begin{minipage}[t]{9cm}
Sei $X$ eine Funktion auf $\Omega$, und lasse sich $\Omega$ in endlich viele
Ereignisse $A_i$ zerlegen, auf denen $X(\omega)$ konstant ist, dann ist der
Erwartungswert von $X$ 
\[ \text{Erwartungswert} = \sum \text{Wert} \cdot \text{Wahrscheinlichkeit}\] 
\[\boxed{E(X)=\sum\limits_{i=0}^n \underbrace{X(A_i)}_{\text{Wert}}\cdot
\underbrace{P(A_i)}_{\text{W'keit}}}\] 
\[E(X) = \int\limits_{-\infty}^x x \cdot \varphi(x) dx$ mit $\varphi(x) =
\text{Dichtefunktion}\] 
\end{minipage}
\hspace{1cm}
\begin{minipage}[t]{9cm}
\subsubsection{Rechenregeln \skript{67}}
\begin{tabular}{ll}
  $E(X+Y)=E(X)+E(Y)$ \\
  $E(\lambda X + \mu) = \lambda \cdot E(X) + \mu$ & $\lambda, \mu \in
  \mathbb{R}$ \\
  $E(XY) = E(X)\cdot E(Y)$ & wenn X,Y unabhängig sind \\
\end{tabular}
\end{minipage}

\hrule


\begin{minipage}{9cm}
\subsection{Varianz \skript{63}}
  \begin{tabular}{ll}
    $\boxed{var(X)=E(X^2)-E(X)^2}= \:E[(X-E(X))^2]$\\
   Standardabweichung $\sigma = \sqrt{var(X)}$
	\end{tabular}
	\subsubsection{Kovarianz}
	\begin{tabular}{ll}
	  $cov(X,Y)=E(XY)-E(X)E(Y)=\underbrace{0}_{\text{falls X,Y unabhängig}}$
	\end{tabular}
\end{minipage}
\begin{minipage}{9cm}
  \subsubsection{Rechenregeln}
  \begin{tabular}{ll}
    $var(\lambda X)=\lambda^2 var(X) \qquad $ $\lambda, \mu
    \in \mathbb{R}$\\
    $var(X_1+X_2+\ldots+X_n) \neq var(n X)$ \\
    $var(X+Y)= \begin{cases}
      var(X)+var(Y)                      &	\text{(X,Y unabh.)}\\                     
      var(X) + var(Y) + 2 \cdot cov(X,Y) &	\text{(X,Y abhängig)}\\
    \end{cases} $ \\
    $var(X Y)= var(Y)var(X)+var(Y)E(X)^2+var(X)E(Y)^2$ \\
  \end{tabular}
\end{minipage}

\hrule

\subsection{Erwartungswert und Varianz des arithmetischen Mittels \skript{78}}
Es sei eine Folge von unabhängigen Zufallsvariablen $X_1, X_2, \ldots , X_n$ mit
gleichem Erwartungswert $ \mu $ und gleicher Varianz $ \sigma^2 $ gegeben. \\
\begin{tabular}{p{6cm} p{6cm} p{6cm}}
  Mittelwert: $M_n=\frac{X_1+\ldots+X_n}{n}$ &
  Erwartungswert: $E(X)=E(M_n) = \mu$  &
  Varianz: $var(M_n)=\frac{1}{n}var(X) = \frac{\sigma ^2}{n} $
\end{tabular}
\vspace{1mm}
\hrule

\subsection{Satz von Tschebyscheff \skript{77}}
\begin{tabular}{ll}
  $P(\left| X-E(X) \right|>\varepsilon)\leq\dfrac{var(X)}{\varepsilon^2}$ &
  Wahrscheinlichkeit, dass $X$ um mehr als $\varepsilon$ vom Erwartungswert $E(X)$ abweicht.\\
  $P(|M_{n}-\mu|>\varepsilon)\leq \frac{\sigma^{2}}{\varepsilon^{2}n} $ &
  W'keit, dass $M_{n}$ von $n$ unab. ZV mit Mittelwert $\mu$ und Varianz $\sigma^{2}$ mehr als $\varepsilon$ von $\mu$ abweicht.
\end{tabular}


\begin{minipage}[t]{9cm}
  \subsection{Regression \skript{80}}
  \begin{tabular}{ll}
    Allgemein: & X,Y Zufallsvariable \\
    Gesucht: & Regressionsgerade $y=ax+b$ mit min. Fehler \\
    Fehler: & $E(Y-(aX+b))=0$ \\
  \end{tabular} \\
 
  \textbf{Regressionskoeffizient r} \\
  $r$ ist ein Mass für die Qualität der Regression (standardisiert) \\
  $r^2=\dfrac{cov(X,Y)^2}{var(X)var(Y)}=a^2\cdot\dfrac{var(X)}{var(Y)}$ \\
 
  \textbf{Mittlerer quadratischer Fehler} \\
  $\Delta^2 = var(Y)(1-r^2) =
  var(Y)\left(1-\dfrac{cov(X,Y)^2}{var(X)var(Y)}\right) $ \\
\end{minipage}
\begin{minipage}[t]{10cm}
  \textbf{Vorgehen:}
  \textcolor{blue}{mit Fehlerberechnung}
	\begin{enumerate}
		\item Tabelle mit bekannten Werten aufstellen:\\
  		\begin{tabular}{|l||l|l||l|l||l|}
  		  \hline
        \textbf{$k$} & \textbf{$x$} & \textbf{$x^2$} & \textbf{$y$} &
  		  \textcolor{blue}{\textbf{$y^2$}} & \textbf{$xy$} \\
  		  \hline \hline
  		  $1$ & $x_1$ & $x_1^2$ & $y_1$ & \textcolor{blue}{$y_1^2$} & $x_1y_1$ \\
  		  \hline
  		  $\vdots$ & $\vdots$ & $\vdots$ & $\vdots$ & \textcolor{blue}{$\vdots$} &
  		  $\vdots$ \\\hline $n$ & $x_n$ & $y_n^2$ & $y_n$ & \textcolor{blue}{$y_n^2$} & $x_ny_n$ \\
  		  \hline
  		  \hline
  		  $\sum$ & $\sum x_k$ & $\sum x_k^2$ & $\sum y_k$ & \textcolor{blue}{$\sum
  		  y_k^2$} & $\sum x_ky_k$ \\
  		  \hline $E$ & $\frac{\sum x_k}{n}$ & $\frac{\sum x_k^2}{n}$ & $\frac{\sum
  		  y_k}{n}$ & \textcolor{blue}{$\frac{\sum y_k^2}{n}$} & $\frac{\sum x_ky_k}{n}$ \\
  		  \hline
  		\end{tabular} 
		\item Varianzen, Kovarianz berechnen: \\
		  $var(X) = E(X^2) - E^2(X)$ \\
		  \textcolor{blue}{$var(Y) = E(Y^2) - E^2(Y)$} \\
		  $cov(X,Y) = E(XY) - E(X)E(Y)$
		\item Koeffizienten \textcolor{blue}{und Fehler} der Gerade berechnen: \\
		  $a=\dfrac{cov(X,Y)}{var(X)}$
		  \textcolor{blue}{$\qquad\Delta^2=var(Y)\left(1-\dfrac{cov(X,Y)^2}
		  {var(X)var(Y)}\right) $} \\
		  $b=E(Y)-aE(X)$
		\item Gerade: \\
		$y=ax+b$
	\end{enumerate}
\end{minipage}

\hrule
